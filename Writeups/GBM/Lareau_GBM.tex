% !TEX TS-program = LuaLaTeX

\documentclass[12pt]{article}

\usepackage[urw-garamond]{mathdesign}
\usepackage[T1]{fontenc}
\usepackage{fontawesome}
\usepackage{enumitem}
\usepackage{scrextend}
\usepackage{etaremune}

% use http link for email
\usepackage[colorlinks = true,
          linkcolor = MidnightBlue,
           urlcolor  = MidnightBlue,
           citecolor = MidnightBlue,
           anchorcolor = MidnightBlue]{hyperref}            
            
%allow formula formatting
\usepackage{amsmath}
\DeclareMathAlphabet\mathbfcal{OMS}{cmsy}{b}{n}
\usepackage{comment}

%title positon
\usepackage{titling} %fix title
\setlength{\droptitle}{-6em}   % Move up the title 

%change section title font size
\usepackage{titlesec}
\titleformat{\section}
  {\normalfont\fontsize{12}{15}}{\thesection}{1em}{}[{\titlerule[0.5pt]}]
  
\titleformat{\subsection}
  {\normalfont\fontsize{12}{13}}{\thesection}{1em}{}

% change page margin and cancel identation
\usepackage[top=0.8in, bottom=0.8in, left=0.7in, right=0.7in]{geometry} 
\setlength{\parindent}{0pt}

%allow modifying section titles 
\usepackage[dvipsnames]{xcolor}

% Make lists without bullets
\renewenvironment{itemize}{
  \begin{list}{}{
    \setlength{\leftmargin}{1.5em}
    \setlength{\itemsep}{0in}
  }
}{
  \end{list}
}

\begin{document}

\begin{minipage}[t]{0.5\linewidth}
    \begin{flushleft}
    \textbf{
    {\huge C}{\Large ALEB} \hspace{-0.1em} 
    {\huge L}{\Large AREAU}}\\
     \hspace*{4mm} BST 290 Advanced Computational Biology \\
     \hspace*{4mm} Prepared 26 October 2016
         \end{flushleft}
\end{minipage}

 
 %% Research Summary Paragraph
 
\section*{\textbf{{\Large S}{\small ECONDARY} {\Large S}{\small UMMARY}{\Large --G}{\small BM}}}
\begin{addmargin}[6.5mm]{4mm}  
To compliment the TCGA GBM paper, I discussed a recent paper by Will Flavahan, Brad Bernstein, and colleagues that provides a mechanism for oncogene activation through the disruption of insulator boundaries. The primary paper showed that the proneural subtype of GBMs could be characterized by a high number (30 percent of cases) of IDH1 mutations, which in part defined a clinical subtype of this brain cancer. Unlike other hallmark genes that differentiate the GBM classes, IDH1 plays a role in the Citric Acid Cycle. Mutations in these genes often render cells non-viable, so the mechanism for this mutation was largely unknown. In the discussion of the primary paper, the authors note that frequent IDH1 mutations are often observed with an amplified expression of PDGFRA, a better characterized oncogene. The secondary manuscript that I presented provided a mechanism that linked IDH1 and PDGFRA through disrupted cancer topology.  \newline \newline The mechanism of activation is as follows: 1) Mutations in IDH causes the abnormal synthesis and accumulation of 2-Hydroxyglutarate, which is slightly different biochemically than alpha-keto-glutarate, a normal by product of the Citric Acid Cycle. 2) The buildup of 2-Hydroxyglutarate inhibits DNA methyltransferase. 3) Consequently, many location genome-wide in cancers with the IDH1 mutation have distinctive patterns of demethylation that result from the inhibition of this critical enzyme that modulates this epigenetic memory, including a CTCF locus that insulates PDGFRA from an enhancer. 4) This insulator is disrupted, the enhancer is localized to the promoter of PDGFRA, and the gene is activated. \newline \newline To achieve this insight, the authors profiled primary tissue of IDH1 mutant and non-IDH1 mutant GBMs and performed differential expression and ChIP-Seq analyses. They identified PDGFRA as a candidate that could be affected by a disrupted boundary due to the difference in correlation of a nearby gene as well as differential expression analyses between the IDH1 mutant statuses. Overall, the validate the topological alteration from 3C experiments as well as a CRISPR alteration on the insulator. \newline \newline 

\end{addmargin}
\end{document}

