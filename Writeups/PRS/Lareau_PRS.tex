% !TEX TS-program = LuaLaTeX

\documentclass[12pt]{article}

\usepackage[urw-garamond]{mathdesign}
\usepackage[T1]{fontenc}
\usepackage{fontawesome}
\usepackage{enumitem}
\usepackage{scrextend}
\usepackage{etaremune}

% use http link for email
\usepackage[colorlinks = true,
          linkcolor = MidnightBlue,
           urlcolor  = MidnightBlue,
           citecolor = MidnightBlue,
           anchorcolor = MidnightBlue]{hyperref}            
            
%allow formula formatting
\usepackage{amsmath}
\DeclareMathAlphabet\mathbfcal{OMS}{cmsy}{b}{n}
\usepackage{comment}

%title positon
\usepackage{titling} %fix title
\setlength{\droptitle}{-6em}   % Move up the title 

%change section title font size
\usepackage{titlesec}
\titleformat{\section}
  {\normalfont\fontsize{12}{15}}{\thesection}{1em}{}[{\titlerule[0.5pt]}]
  
\titleformat{\subsection}
  {\normalfont\fontsize{12}{13}}{\thesection}{1em}{}

% change page margin and cancel identation
\usepackage[top=0.8in, bottom=0.8in, left=0.7in, right=0.7in]{geometry} 
\setlength{\parindent}{0pt}

%allow modifying section titles 
\usepackage[dvipsnames]{xcolor}

% Make lists without bullets
\renewenvironment{itemize}{
  \begin{list}{}{
    \setlength{\leftmargin}{1.5em}
    \setlength{\itemsep}{0in}
  }
}{
  \end{list}
}

\begin{document}

\begin{minipage}[t]{0.5\linewidth}
    \begin{flushleft}
    \textbf{
    {\huge C}{\Large ALEB} \hspace{-0.1em} 
    {\huge L}{\Large AREAU}}\\
     \hspace*{4mm} BST 290 Advanced Computational Biology \\
     \hspace*{4mm} Prepared 16 November 2016
         \end{flushleft}
\end{minipage}


\section*{\textbf{{\Large S}{\small ECONDARY} {\Large S}{\small UMMARY}{\Large --G}{\small WAS OF MANY }{\Large D}{\small ISEASES} }}
\begin{addmargin}[6.5mm]{4mm}
To compliment that primary paper looking at risk variants across many disorders, I presented on the concept of polygenic risk scoring by highlighting an application from a Nature Neuroscience paper where this methodology related schizophrenia, bipolar disorder, and creativity. In general, I thought of this methodology to be useful to functionalize some of the summary statistics that have resulted from the GWAS era. As the methodology to compute polygenic risk scores is fairly straightforward and many tools such as PRSice (Eusden et al. 2015) exist, I'd imagine that they become routinely added in modern genetic association studies. Moreover, the potential therapeutic value of these predisposition scores further broadens their modern utility. \newline \newline In the present study, Power et al. uses external summary statistics from bipolar disorder and schizophrenia to predict a variety of traits from 86,000+ individuals from an Iceland genotyping study. Of note, the authors use the professional status of these individuals to define the "creative" phenotype, notably which excludes scientists. The class discussion noted that this isn't the best phenotypic definition of creative, which is seemingly valid criticism. Moreover, for the 800+ individuals where a quantitative creative phenotype was available, the PRS models for bipolar and schizophrenia was not significant but some replication within the professional statuses of these secondary cohorts suggested that the relationship between creative professions and genetic predisposition to bipolar disorder and schizophrenia may persist. The summary tables support that this association may be valid as other mechanisms of slicing the data fails to yield statistically significant models as the creativity association that the authors introduce in the two figures and table. \newline \newline Overall, the discussion from the class seemed to suggest that this paper was clever and 'cute' but not particularly robust due to the necessarily hazy phenotypic definition of creativity. However, the general framework is useful to consider how traits can be genetically correlated, which provides potentially useful subcategories of disease states. Polygenic risk scoring is a good methodology for inferring individual-level risk scores while newer methods such as LD Score Regression can infer correlations broadly between disorders to determine relatedness. I would suggest that classifying previously binary phenotypes into continuous subtypes based on genetic risk may provide an efficacious means to combat the problem of missing heritability.  
\end{addmargin}
\end{document}

